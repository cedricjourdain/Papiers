\documentclass[11pt]{lettre}
\usepackage[utf8]{inputenc}
\usepackage[T1]{fontenc}
\usepackage[francais]{babel}
\usepackage{lmodern}
 
\lieu{}
\begin{document}
	\begin{letter}{SCI Bauval\\ Le Boulidou,\\ 34400, Verargues}
		\signature{Cédric \textsc{Jourdain}}
		\address{9, rue boussinesq}
		\telephone{0678980130}
		\email{cedric.jourdain56@gmail.com}
		\conc{Mise en demeure de réaliser les réparations nécessaires dans le logement \\ Recommandé avec A. R.}
		\opening{Madame, monsieur,}
 		%https://location-immobilier.ooreka.fr/ebibliotheque/voir/184596/mise-en-demeure-a-un-bailleur-de-faire-effectuer-les-reparations-necessaires-dans-le-logement-loue-p
		
	
		J’occupe un logement que vous me louez au 9, rue boussinesq, depuis le 12 janvier 2018

		Malgré mes différents appels téléphoniques, nos entevues et proposition de devis (cf. devis de MB-plomberie 06/06/2018), aucune action visible de votre part n'a été faite. Je vous rappelle que j'ai constaté dans le logement que vous me louez, (et ce, depuis mon arrivé dans le logement en javier 2018), suivant le bail signé le 12 janvier 2018, un certain nombre de dysfonctionnements qui nécessitent les travaux suivants : rénovation de la salle de bain (Extraction du bac à douche et mise en place d'un nouveau, pose d'un receveur, pose d 'une couche de peinture hydrofugé,Pose de carrelage murale), changement de la porte d'entrée plus que dégradée.

		Je vous demande donc de la réaliser le plus rapidement possible. En outre, si les nuisances persistaient plus de 40 jours, je serai en droit d’exiger une baisse de loyer pour réparer le préjudice.
 
		\closing{Dans l’attente d’une réponse de votre part que j’espère favorable, je vous prie d'agréer, Madame, Monsieur, mes salutations distinguées. }
 
	\end{letter}
\end{document}