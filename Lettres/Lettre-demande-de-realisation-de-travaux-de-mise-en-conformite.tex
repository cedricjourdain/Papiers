\documentclass[11pt]{lettre}
\usepackage[utf8]{inputenc}
\usepackage[T1]{fontenc}
\usepackage[francais]{babel}
\usepackage{lmodern}
 
\begin{document}
	\begin{letter}{SCI Bauval\\ Le Boulidou,\\ 34400, Verargues}
		\signature{Cédric \textsc{Jourdain}, Cora \textsc{Perroud}, Marianne \textsc{Vidal}, Sylvain\textsc{Perico}}
		\address{9, rue boussinesq}
		\lieu{Montpellier}
		\telephone{0678980130}
		\email{cedric.jourdain56@gmail.com}
		\conc{Demande de réaliser les travaux de mise en conforité du logement\\ Recommandé avec A. R.}
		\opening{Madame, monsieur,}
 		%exemple : https://location-immobilier.ooreka.fr/ebibliotheque/voir/122511/demande-de-mise-aux-normes-d-un-logement-en-location
%https://www.lebonbail.fr/modeles-de-documents/demande-au-proprietaire-de-realiser-des-travaux-de-mise-aux-normes
		Nous  avons signé un contrat de bail le 12 janvier pour le logement situé au 9, rue Boussinesq, 34070 Montpellier.

		Le logement présente des dégradations qui nécessitent des réparations qui n'entrent pas dans le champ des réparations locatives à la charge du locataire, définies par le décret du 26 août 1987.
		
		En effet, le logement présente les problèmes suivantes : Salle de bain insalubre, porte d'entrée dans un état dégradé.
		
		Malgré nos différents appels téléphoniques, nos entevues et proposition de devis (cf. devis de MB-plomberie 06/06/2018), et ce depuis la signature du bail en janvier 2018, aucune action visible de votre part n'a été faite. 

		Il est de mon devoir de vous avertir de ces dommages afin que vous puissiez rapidement agir et éviter une aggravation de la situation. De votre côté, vous êtes en effet tenu d’effectuer « toutes les réparations, autres que locatives, nécessaires au maintien en état et à l’entretien normal des locaux loués » (article 6 de la loi du 6 juillet 1989). Par conséquent, je vous prie d'effectuer les travaux et réparations nécessaires, et ce dans un délai de 1 mois à partir d'aujourd'hui. 

 		\closing{Dans l’attente d’une réponse de votre part que j’espère favorable, je vous prie d'agréer, Madame, Monsieur, mes salutations distinguées.}
 
	\end{letter}
\end{document}