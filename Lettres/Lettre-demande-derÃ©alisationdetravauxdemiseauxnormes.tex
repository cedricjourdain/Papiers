\documentclass[11pt]{lettre}
\usepackage[utf8]{inputenc}
\usepackage[T1]{fontenc}
\usepackage[francais]{babel}
\usepackage{lmodern}
 
\begin{document}
	\begin{letter}{Adresse du destinataire}
		\signature{Cédric \textsc{Jourdain}}
		\address{9, rue boussinesq}

		\telephone{0678980130}
		\email{cedric.jourdain56@gmail.com}
		\conc{demande de réalisation de  travaux de mise aux norme\\ Recommandé avec A. R.}
		\opening{Madame, monsieur,}
 		%exemple : https://location-immobilier.ooreka.fr/ebibliotheque/voir/122511/demande-de-mise-aux-normes-d-un-logement-en-location
		Nous  avons signé un contrat de bail le 12 janvier pour le logement situé au 9, rue Boussinesq, 34070 Montpellier.

		Malheureusement,  le logement que je vous loue ne respecte pas les normes d'un logement décent définies par le décret n 2002-120  de janvier 2002 modifié par le décret n 2017-312 du 9 mars 2017. En particulier, \\% [détailler les manquements].
""Article 2 \\
4. La nature et l'état de conservation et d'entretien des matériaux de construction, des canalisations et des revêtements du logement ne présentent pas de risques manifestes pour la santé et la sécurité physique des locataires ;\\
6. Le logement permet une aération suffisante. Les dispositifs d'ouverture et les éventuels dispositifs de ventilation des logements sont en bon état et permettent un renouvellement de l'air et une évacuation de l'humidité adaptés aux besoins d'une occupation normale du logement et au fonctionnement des équipements ;"

En conséquence, je vous sollicite afin que vous réalisiez, à vos frais, les travaux de mise aux normes qui s'imposent. À défaut d’une réponse adéquate de votre
 part dans un délai de 30 jours à compter de la remise de la présente lettre, je me verrai dans l’obligation de saisir les autorités compétentes.
 
		\closing{Je vous prie d’agréer, Madame, Monsieur, mes salutations distinguées.}
 
	\end{letter}
\end{document}